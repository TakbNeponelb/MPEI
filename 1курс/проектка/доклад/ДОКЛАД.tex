\documentclass[a4paper]{article}
\usepackage[14pt]{extsizes} % для того чтобы задать нестандартный 14-ый размер шрифта
\usepackage[utf8]{inputenc}
\usepackage[russian]{babel}
\usepackage{setspace,amsmath}
\usepackage{epigraph} % для эпиграфов и продвинутых цитат
\usepackage{csquotes} % ещё одна штука для цитат
\usepackage[unicode, pdftex]{hyperref} % подключаем hyperref (для ссылок внутри  pdf)
\usepackage{amssymb} % в том числе для красивого знака пустого множества
\usepackage{amsthm} % в т.ч. для оформления доказательств
\usepackage[left=20mm, top=15mm, right=15mm, bottom=15mm, nohead, footskip=10mm]{geometry} % настройки полей документа
 
\usepackage[active]{srcltx}
\newcommand{\ran}{{\rm ran}\,}
\newcommand{\diag}{{\rm diag}\,}
% переименовываем  список литературы в "список используемой литературы"
\addto\captionsrussian{\def\refname{Список используемой литературы}}
\newcounter{totreferences}
\pretocmd{\bibitem}{\addtocounter{totreferences}{1}}{}{}
\newtheorem{theorem}{Теорема} % задаём выводимое слово (для теорем)
\newtheorem{definition}{Опредление} % задаём выводимое слово (для определений)
 
% объявляем новые команды
 
% новая команда \RNumb для вывода римских цифр
\newcommand{\RNumb}[1]{\uppercase\expandafter{\romannumeral #1\relax}}



\begin{document}

\begin{titlepage}

\thispagestyle{empty}

\centerline{ФЕДЕРАЛЬНОЕ ГОСУДАРСТВЕННОЕ БЮДЖЕТНОЕ ОБРАЗОВАТЕЛЬНОЕ}
\centerline{УЧРЕЖДЕНИЕ ВЫСШЕГО ОБРАЗОВАНИЯ}
\centerline{"НАЦИОНАЛЬНЫЙ ИССЛЕДОВАТЕЛЬСКИЙ УНИВЕРСИТЕТ "МЭИ"}

\vfill

\begin{center}
\Large Институт Информационных и Вычислительных Технологий (ИВТИ) \\ 
\end{center}

\vfill

\centerline{\huge{Доклад}}
\centerline{\large{по дисциплине}}
\centerline{\LARGE{Проектная деятельность}}

\vfill

\begin{center}
\textsc{\textbf{Понятие биоритмов }}
\end{center}

\vfill

Студент группы А-07-22 \hfill Татарников Максим Станиславович

Преподаватель \hfill Вовк Марина Витальевна

\vfill

\centerline{Москва, 2023}
\clearpage
\end{titlepage}

\input{Основная часть.\TeX}
\newpage
\tableofcontents
\end{document}