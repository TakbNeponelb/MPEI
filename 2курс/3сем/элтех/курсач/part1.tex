\newpage

\section{Расчет установившихся режимов линейной электрической цепи при постоянных воздействиях}


При постоянном токе конденсатор создает разрыв, катушка индуктивности закорочена, источник тока и источник напряжения имеют только компоненты $J_0, E_0$ соответственно, поэтому возможно перерисовать цепь следующим образом:

\begin{figure}[h]
    \centering
    \begin{circuitikz}[scale = 0.85]
    \ctikzset{inductors/scale=1.5, inductor=american}
        \draw (0, 0) -- ++(0,1) to[short, *-*] ++(0,2) --  ++(0,1) to[R, l=$R_1$, european] ++(0,4) to[american current source, l=$E_0$, -*] ++(4,0) coordinate(A);
        \draw (A) to[short, -*] ++(0,-1.5);
        \draw (4,4) to[R, l=$R_3$, european, -*] ++(0,-4) coordinate(C) -- ++(-4,0);
        \draw (4,4) to[short, -*] ++(0,1.5);
        \draw (A) -- ++(4,0) coordinate(D);
        \draw (D) -- ++(4,0) to[R, european, l=$R_6$, -*] ++(0,-4) coordinate(E);
        \draw (D) to[R, *-*, european, l=$R_5$] ++(0,-4) coordinate(F) to[R, european, -*, l=$R_2$] ++(0,-4);
        \draw (4,4) to[R, european, l=$R_H$] (F) -- (E);
        \draw (C) -- ++(8,0) to[esource, l=$J_0$, name = I] (E);
        \node at (I)[rotate=90]{> >};
        
    \end{circuitikz}
    \caption{Электрическая цепь}
    \label{fig:dc_circuit}
\end{figure}

\subsection{Нахождение потенциалов и токов методом узловых потенциалов}

Расставим потенциалы, случайным образом укажем направления токов в ветвях:

\begin{figure}[h]
    \centering
    \begin{circuitikz}[scale = 0.85]
    \ctikzset{inductors/scale=1.5, inductor=american}
        \draw (0, 0) -- ++(0,4) to[R, l=$R_1$, european] ++(0,4) to[american current source, l=$E_0$, i=$I_1$] ++(4,0) coordinate(A);
        \draw (4,4) to[R, l=$R_3$, european, -*, i=$I_3$] ++(0,-4) coordinate(C) -- ++(-4,0);
        \draw (A) -- ++(4,0) coordinate(D);
        \draw (D) -- ++(4,0) to[R, european, l=$R_6$, -*, i=$I_6$] ++(0,-4) coordinate(E);
        \draw (D) to[R, *-*, european, l=$R_5$, i=$I_5$] ++(0,-4) coordinate(F) to[R, european, -*, l=$R_2$, i=$I_2$] ++(0,-4) coordinate(H);
        \draw (4,4) to[R, european, l=$R_H$] (F) -- (E);
        \draw (C) -- ++(8,0) coordinate(G);
        \draw (E) to[esource, l=$J_0$, name = I, i=$J$] (G);
        \node at (I)[rotate=90]{> >};
        \node at (D)[yshift = 3mm]{$\varphi_1$};
        \node at (F)[yshift = -3mm, xshift = 3mm]{$\varphi_2$};
        \node at (H)[yshift = -3mm]{$\varphi_0=0~B$};
        
        
    \end{circuitikz}
    \caption{Электрическая цепь}
    \label{fig:dc_potencial_circuit}
\end{figure}

Определим потенциалы узлов и узловые токи в ветвях с помощью метода узловых потенциалов, при условии, что $\varphi_0 = 0~B$:

Рассчитаем проводимости каждой ветви:
\\
$G_1 = \frac{1}{R_1} = \frac{1}{179} = 5.59$ мСм
\\
$G_2 = \frac{1}{R_2} = \frac{1}{250} = 4$ мСм
\\
$G_3 = \frac{1}{R_3 + R_H} = \frac{1}{229 + 235} = 2.16$ мСм
\\
$G_5 = \frac{1}{R_5} = \frac{1}{212} = 4.72$ мСм
\\
$G_6 = \frac{1}{R_6} = \frac{1}{212} = 4.72$ мСм
\\
$G_J = 0$ См
\\

Рассчитаем собственные и взаимные проводимости:
\\
$G_{11} = G_1 + G_5 + G_6 = 5.59 + 4.72 + 4.72 = 15.03$ мСм
\\
$G_{22} = G_2 + G_3 + G_5 + G_6 = 4 + 2.16 + 4.72 + 4.72 = 15.6$ мСм
\\
$G_{12} = G_{21} = G_5 + G_6 = 4.72 + 4.72 = 9.44$ мСм
\\

Узловые токи:
\\
$J^y_1 = \frac{E_0}{R_1} = \frac{10.3}{179} = 57.5$ мА
\\
$J^y_2 = J = J_0 = 140$ мА
\\

Запишем уравнения:
\\
$
\begin{cases}
\varphi_1G_{11} - \varphi_2G_{12} = J^y_1
\\
-\varphi_1G_{21} + \varphi_2G_{22} = J^y_2
\end{cases}
$
\\

Подставим значения:
\\
$
\begin{cases}
15.03\varphi_1 - 9.44\varphi_2 = 57.5
\\
-9.44\varphi_1 + 15.6\varphi_2 = 140
\end{cases}
$
\\

Составим Матричное уравнение:
\\
\\
$
\begin{pmatrix}
    15.03 & -9.44\\
    -9.44 & 15.6 
\end{pmatrix}
\cdot
\begin{pmatrix}
    \varphi_1 \\
    \varphi_2
\end{pmatrix}
=
\begin{pmatrix}
    57.5 \\
    140
\end{pmatrix}
$
\\

Решив матрицу, получаем ответ:
\\
$\begin{cases}
\varphi_1 = 15.26~B
\\
\varphi_2 = 18.21~B
\end{cases}$
\\

Рассчитаем узловые токи:
\\
$\begin{cases}
I_1 = (\varphi_0 - \varphi_1 + E_0) \cdot G_1 = (0 - 15.26 + 10.3) \cdot 5.59 = -27.73 ~\text{мА}
\\
I_2 = (\varphi_2 - \varphi_0) \cdot G_2 = (18.21 - 0 ) \cdot 4 = 72.84 ~\text{мА}
\\
I_3 = (\varphi_2 - \varphi_0) \cdot G_3 = (18.21 - 0 ) \cdot 2.16 = 39.33 ~\text{мА}
\\
I_5 = (\varphi_1 - \varphi_2) \cdot G_5 = (15.26 - 18.21) \cdot 4.72 = -13.92 ~\text{мА}
\\
I_6 = (\varphi_1 - \varphi_2) \cdot G_6 = (15.26 - 18.21) \cdot 4.72 = -13.92 ~\text{мА}
\end{cases}$

Знак минус перед токами $I_1$, $I_5$ и $I_6$ обозначает, что истинное направление токов противоположно.

\newpage
\subsection{Расчет параметров двухполюсника $(U_p, R_{BX}, I_{\text{кз}})$ относительно сопротивления $R_H$}

Перерисуем схему относительно $R_H$ для нахождения эквивалентного сопротивления и напряжения:

\begin{figure}[h]
    \centering
    \begin{circuitikz}[scale = 0.85]
    \ctikzset{inductors/scale=1.5, inductor=american}
        \draw (0, 0) -- ++(0,4) to[R, l=$R_1$, european] ++(0,4) to[american current source, l=$E_0$] ++(4,0) coordinate(A);
        \draw (4,4) to[R, l=$R_3$, european, -*] ++(0,-4) coordinate(C) -- ++(-4,0);
        \draw (A) -- ++(4,0) coordinate(D);
        \draw (D) -- ++(4,0) to[R, european, l=$R_6$, -*] ++(0,-4) coordinate(E);
        \draw (D) to[R, *-*, european, l=$R_5$] ++(0,-4) coordinate(F) to[R, european, -*, l=$R_2$] ++(0,-4) coordinate(H);
        
        \draw (4,4) to[short, -*] (5.5,4) ;
        \draw (6.5,4) to[short, *-](F) -- (E);
        \node at (6,4)[yshift = 3mm]{$R_{BX}$};
        \draw (C) -- ++(8,0) coordinate(G);
        \draw (E) to[esource, l=$J_0$, name = I] (G);
        \node at (I)[rotate=90]{> >};
        
    \end{circuitikz}
    \caption{Электрическая цепь}
    \label{fig:dc_equ_circuit}
\end{figure}

Закоротим источник напряжения и разорвем цепь на месте источника тока:

\begin{figure}[h]
    \centering
    \begin{circuitikz}[scale = 0.85]
    \ctikzset{inductors/scale=1.5, inductor=american}
        \draw (0, 0) -- ++(0,4) to[R, l=$R_1$, european] ++(0,4) -- ++(4,0) coordinate(A);
        \draw (4,4) to[R, l=$R_3$, european, -*] ++(0,-4) coordinate(C) -- ++(-4,0);
        \draw (A) -- ++(4,0) coordinate(D);
        \draw (D) -- ++(4,0) to[R, european, l=$R_6$, -*] ++(0,-4) coordinate(E);
        \draw (D) to[R, *-*, european, l=$R_5$] ++(0,-4) coordinate(F) to[R, european, -*, l=$R_2$] ++(0,-4) coordinate(H);
        
        \draw (4,4) to[short, -*] (5.5,4) ;
        \draw (6.5,4) to[short, *-](F) -- (E);
        \node at (6,4)[yshift = 3mm]{$R_{BX}$};
        \draw (C) -- ++(4,0) coordinate(G);
        
    \end{circuitikz}
    \caption{Электрическая цепь}
    \label{fig:dc_equ_circuit}
\end{figure}

\newpage
Перерисуем схему для более удобного вычисления $R_{BX}$:

\begin{figure}[h]
    \centering
    \begin{circuitikz}[scale = 0.85]
        
        \draw (0,0) to[R, l=$R_3$, european, *-] ++(4,0) coordinate(C);

        \draw (C) -| ++(0,2) to[R, l=$R_1$, european] ++(4,0)coordinate(A);
        \draw (C) -| ++(0,-2) to[R, l=$R_2$, european] ++(9,0) -- ++(0,2) to[short, -*] ++(2,0);
        \draw (A) -| ++(0,1) to[R, l=$R_5$, european] ++(4,0) |- ++(1,-1) -- ++(0,-2);
        \draw (A) -| ++(0,-1) to[R, l=$R_6$, european] ++(4,0) |- ++(0,1);
        
    \end{circuitikz}
    \caption{Общее сопротивление}
    \label{fig:dc_res_circuit}
\end{figure}

Рассчитаем общее сопротивление $R_\text{вх}$:
\\
$R_\text{вх} = R_3 + R_{1256} = 229 + 133.18 = 362.18$ Ом
\\
$R_{1256} = \frac{R_{156}\cdot R_2}{R_2 + R_{156}} = \frac{285 \cdot 250}{285 + 250} = 133.18$ Ом
\\
$R_{156} = R_1 + R_{56} = 179 + 106 = 285$ Ом
\\
$R_{56} = \frac{R_5 \cdot R_6}{R_5 + R_6} = \frac{212 \cdot 212}{212 + 212} = 106$ Ом
\\

Определим $U_p$ с помощью метода суперпозиции:

\begin{figure}[h]
    \centering
    \begin{circuitikz}[scale = 0.8]
    \ctikzset{inductors/scale=1.5, inductor=american}
        \draw (0, 0) -- ++(0,4) to[R, l=$R_1$, european] ++(0,4) to[american current source, l=$E_0$] ++(4,0) coordinate(A);
        \draw (4,4) to[R, l=$R_3$, european, -*] ++(0,-4) coordinate(C) -- ++(-4,0);
        \draw (A) -- ++(4,0) coordinate(D);
        \draw (D) -- ++(4,0) to[R, european, l=$R_6$, -*] ++(0,-4) coordinate(E);
        \draw (D) to[R, *-*, european, l=$R_5$] ++(0,-4) coordinate(F) to[R, european, -*, l=$R_2$] ++(0,-4) coordinate(H);
        
        \draw (4,4) to[short, -*] (5.5,4) ;
        \draw (6.5,4) to[short, *-](F) -- (E);
        \ctikzset{voltage=straight}
        \draw (7,4) to[short, v=$U_p$] (5,4);
        \draw[white] (5.6,4) to[short] (6.4,4);
        \draw (C) -- ++(8,0) coordinate(G);
        \draw (E) to[esource, l=$J_0$, name = I] (G);
        \node at (I)[rotate=90]{> >};
        
    \end{circuitikz}
    \caption{Электрическая цепь}
    \label{fig:dc_equ_circuit}
\end{figure}

Для расчета эквивалентного напряжения $U_p$ методом суперпозиции,
отметим в цепи частичные токи и заменим сначала источник тока $J_0$ на разрыв цепи.

\newpage
\begin{figure}[h]
    \centering
    \begin{circuitikz}[scale = 0.8]
    \ctikzset{inductors/scale=1.5, inductor=american}
        \draw (0, 0) -- ++(0,4) to[R, l=$R_1$, i=$I^1_1$, european] ++(0,4) to[american current source, l=$E_0$] ++(4,0) coordinate(A);
        \draw (4,4) to[R, l=$R_3$,i=${I^1_3 = I^1_p}$, european, -*] ++(0,-4) coordinate(C) -- ++(-4,0);
        \draw (A) -- ++(4,0) coordinate(D);
        \draw (D) -- ++(4,0) to[R, european, l=$R_6$,i=$I^1_6$, -*] ++(0,-4) coordinate(E);
        \draw (D) to[R, *-*, european, l=$R_5$, i=$I^1_5$] ++(0,-4) coordinate(F) to[R, european, -*, l=$R_2$, i=$I^1_2$] ++(0,-4) coordinate(H);
        \draw (4,4) to[short, -*] (5.5,4) ;
        \draw (6.5,4) to[short, *-](F) -- (E);
        \ctikzset{voltage=straight}
        \draw (7,4) to[short, v=$U^1_p$] (5,4);
        \draw[white] (5.6,4) to[short] (6.4,4);
        \draw (C) -- ++(8,0) coordinate(G);
        \draw (G) to[short, -*] ++(0,1.5);
        \draw (E) to[short, -*] ++(0,-1.5);
        
    \end{circuitikz}
    \caption{Электрическая цепь}
    \label{fig:dc_equ_circuit}
\end{figure}

Вместе с $U_p$ последовательно подключен $R_3$, так как в этом месте разрыв цепи, то в расчетах частичных токов оно учитываться не будет.

Свернем $R_5$ и $R_6$ в один резистор $R_{56}$, чтобы получить схему из одной ветви и рассчитать силу тока через закон Ома, причем $I^1_1 = I^1_2 = I^1_{56} = I^1$.
\\ \\
$R_{56} = \frac{R_5 \cdot R_6}{R_5 + R_6} = \frac{212 \cdot 212}{212 + 212} = 106 ~\text{Ом}$
\\

\begin{figure}[h]
    \centering
    \begin{circuitikz}[scale = 0.6]
    \ctikzset{inductors/scale=1.5, inductor=american}
        \draw (0, 0) -- ++(0,4) to[R, l=$R_1$, i=$I^1_1$, european] ++(0,4) to[american current source, l=$E_0$] ++(4,0) coordinate(A);
        %\draw (4,4) to[R, l=$R_3$,i=$I^1_3$, european, -*] ++(0,-4) coordinate(C) -- ++(-4,0);
        \draw (A) -- ++(4,0) coordinate(D);
        %\draw (D) -- ++(4,0) to[R, european, l=$R_6$,i=$I^1_6$] ++(0,-4) coordinate(E);
        \draw (D) to[R, european, l=$R_{56}$, i=$I^1_{56}$] ++(0,-4) coordinate(F) to[R, european, l=$R_2$, i=$I^1_2$] ++(0,-4) coordinate(H);
        \draw (0,0) -- ++(8,0);
        %\draw (4,4) to[short, -*] (5.5,4) ;
        %\draw (6.5,4) to[short, *-](F);
        %\ctikzset{voltage=straight}
        %\draw (7,4) to[short, v=$U^1_p$] (5,4);
        %\draw[white] (5.6,4) to[short] (6.4,4);
        %\draw (C) -- ++(4,0) coordinate(G);
        
    \end{circuitikz}
    \caption{Электрическая цепь}
    \label{fig:dc_equ_circuit}
\end{figure}


$I' = \frac{E_0}{R_1 + R_2 + R_{56}} = \frac{10.3}{179 + 250 + 106} = 19.25 ~\text{мА}$\\

\newpage

Изменим схему, подключив источник тока и закоротив источник напряжения:

\begin{figure}[h]
    \centering
    \begin{circuitikz}[scale = 0.6]
    \ctikzset{inductors/scale=1.5, inductor=american}
        \draw (0, 0) -- ++(0,4) to[R, l=$R_1$, european] ++(0,4) -- ++(4,0) coordinate(A);
        %\draw (4,4) to[R, l=$R_3$, european, -*] ++(0,-4) coordinate(C) -- ++(-4,0);
        \draw (0,0) -- (4,0) coordinate(C);
        \draw (A) -- ++(4,0) coordinate(D);
        \draw (D) -- ++(4,0) to[R, european, l=$R_6$, -*] ++(0,-4) coordinate(E);
        \draw (D) to[R, *-*, european, l=$R_5$] ++(0,-4) coordinate(F) to[R, european, -*, l=$R_2$] ++(0,-4) coordinate(H);
        \draw (F) -- (E);
        %\draw (4,4) to[short, -*] (5.5,4) ;
        %\draw (6.5,4) to[short, *-](F) -- (E);
        %\ctikzset{voltage=straight}
        %\draw (7,4) to[short, v=$U_p$] (5,4);
        %\draw[white] (5.6,4) to[short] (6.4,4);
        \draw (C) -- ++(8,0) coordinate(G);
        \draw (E) to[esource, l=$J_0$, name = I] (G);
        \node at (I)[rotate=90]{> >};
        
    \end{circuitikz}
    \caption{Электрическая цепь}
    \label{fig:dc_equ_circuit}
\end{figure}

Сделаем замену источника тока $J_0$ и резистора $R_2$ на эквивалентный источник напряжения $E_2$ и резистор $R_2$.
\\
$E_2 = J_0 \cdot R_2 = 0.140 \cdot 250 = 35 B$

Преобразовав $R_5$ и $R_6$ в один резистор $R_{56}$, снова получим цепь, состоящую из одной ветви. Направим ток против часовой стрелки, сонаправлено с источником тока, и получим следующую схему:

\begin{figure}[h]
    \centering
    \begin{circuitikz}[scale = 0.6]
        \draw (0, 0) -- ++(0,4) to[R, l=$R_1$, european] ++(0,4) -- ++(4,0) coordinate(A);
        \draw (0,0) -- (4,0) coordinate(C);
        \draw (8,8) coordinate(D) to[short, i=$I''$] (A) ;
        \draw (D) to[R, european, l=$R_{56}$] ++(0,-4) coordinate(F) to[american current source,invert, l=$E_2$]++(0,-4) coordinate(H);
        \draw (C) to[R, l=$R_2$, european] ++(4,0) coordinate(G);
        
    \end{circuitikz}
    \caption{Электрическая цепь}
    \label{fig:dc_equ_circuit}
\end{figure}

По закону Ома:
\\
$I'' = \frac{E_2}{R_1 + R_2 + R_{56}} = \frac{35}{179 + 250 + 106} = 65.42 ~\text{мА}$
\\

Так как токи текут в разные стороны, то вычитая из большего меньший, получаем общий ток, направленный против источника напряжения и сонаправленный с источником тока:
\\
$I = I'' - I' = 65.42 - 19.25 = 46.17~\text{мА}$

\newpage

Возвращаясь к исходной схеме, расставим потенциалы и посчитаем напряжение между ними:

\begin{figure}[h]
    \centering
    \begin{circuitikz}[scale = 0.7]
    \ctikzset{inductors/scale=1.5, inductor=american}
        \draw (0, 0) -- ++(0,4) to[R, l=$R_1$, european] ++(0,4) to[american current source, l=$E_0$] ++(4,0) coordinate(A);
        \draw (4,4) to[R, l=$R_3$, european, -*] ++(0,-4) coordinate(C) -- ++(-4,0);
        \draw (A) -- ++(4,0) coordinate(D);
        \draw (D) -- ++(4,0) to[R, european, l=$R_6$, -*] ++(0,-4) coordinate(E);
        \draw (D) to[R, *-*, european, l=$R_5$] ++(0,-4) coordinate(F) to[R, european, -*, l=$R_2$] ++(0,-4) coordinate(H);
        
        \draw (4,4) to[short, -*] (5.5,4) ;
        \draw (6.5,4) to[short, *-](F) -- (E);
        \ctikzset{voltage=straight}
        \draw (7,4) to[short, v=$U_p$] (5,4);
        \draw[white] (5.6,4) to[short] (6.4,4);
        \draw (C) -- ++(8,0) coordinate(G);
        \draw (E) to[esource, l=$J_0$, name = I] (G);
        \node at (I)[rotate=90]{> >};
        \node at (H)[yshift = -3mm]{c};
        \node at (F)[yshift = -3mm, xshift = 3mm]{b};
        \node at (D)[yshift = 3mm]{a};
        \node
    \end{circuitikz}
    \caption{Электрическая цепь}
    \label{fig:dc_equ}
\end{figure}

Рассмотрим часть цепи между потенциалами $\varphi_b ~\text{и}~ \varphi_c $:
$U_p = U_2 = I_2 \cdot R_2$

По первому закону Кирхгофа $I_2 = J_0 - I_1$, где $I_1 = I = 46.17~\text{мА}$ из метода суперпозиции, тогда:
\\
$I_2 = J_0 - I_1 = 140 - 46.17 = 93.83 ~\text{мА}$
\\
$U_p = U_2 = I_2 \cdot R_2 = 93.83 \cdot 10^{-3}\cdot 250 = 23.46 ~B$
\\
В итоге, можно перерисовать схему с использованием эквивалентного источника напряжения и эквивалентного сопротивления:

\begin{figure}[h]
    \centering
    \begin{circuitikz}
        \draw (0,0) to[american current source, l=${E_\text{экв} = U_p}$] ++(0,4) to[R, l=${R_\text{экв} = R_\text{вх}}$, european] ++(4,0) to[R, l=$R_H$, european] ++(0,-4) -- (0,0);
    \end{circuitikz}
    \caption{Схема эквивалентного генератора}
    \label{fig:enter-label}
\end{figure}

Рассчитаем $I_\text{кз}$ и $I_p$ по определению:
\\
$I_\text{кз} = \frac{U_p}{R_\text{вх}} = \frac{23.45}{362.2} = 64.74~\text{мА}$
\\
$I_H = \frac{U_p}{R_\text{вх} + R_H} = \frac{23.45}{362.2 + 235} = 39.27~\text{мА}$
\\
\\
Ответ: $$U_p = 23.45~B, R_\text{вх} = 362.2~\text{Ом}, I_\text{кз} = 64.74~\text{мА}$$

Сравнивая значения $I_H$, полученные методом узловых потенциалов и методом эквивалентного генератора, разница их значений совпадает на $\frac{I_H}{I_{H\varphi}} = \frac{39.27}{39.33} = 0.998$ или $99.8\%$, то есть погрешность в вычислениях составила менее $0.2\%$.

\subsection{Составление уравнения баланса мощности}

Составим уравнение баланса активных мощностей.

Общая формула:
\\
$\displaystyle\sum P_\text{ист} = \displaystyle\sum P_\text{пр}$ \\ $P_\text{пр} = I^2R;\\P_\text{ист} = UI \\ \displaystyle\sum I\cdot E = \displaystyle\sum I^2 \cdot R$ 
\\
$- E_0 \cdot I_1 + (\varphi_2 - \varphi_0) \cdot J = I_1^2R_1 + I_2^2R_2 + I_3^2(R_3+R_H) + I_5^2R_5 + I_6^2R_6$
\\
$-10.3 \cdot 27.73\cdot 10^{-3} + 18.21 \cdot 140 \cdot 10^{-3} = (27.73 \cdot 10^{-3})^2 \cdot 179 + (72.84\cdot 10^{-3})^2 \cdot 250 + (39.33 \cdot 10^{-3})^2 \cdot (229+235) + (13.92\cdot 10^{-3})^2 \cdot 212 + (13.92\cdot 10^{-3})^2 \cdot 212$\\
$$2.263 ~\text{Вт} = 2.264~\text{Вт}$$

Рассчитаем погрешность вычислений:
\\
$\delta = \frac{|P_\text{ист} - P_\text{пр}|}{P_\text{ист}} \cdot 100\% = \frac{|2.263 - 2.264|}{2.263} \cdot 100\% = 0.038\% < 1\% \implies$ вычисления сошлись, значения верны. 

Так как все вычисления были проделаны два раза, двумя разными методами, а также проверены уравнением баланса мощности, и в итоге различие значений составило менее $1\%$, то можно с уверенностью сказать, что ответы верны.








